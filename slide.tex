\documentclass{beamer}
%\usetheme{PaloAlto}
%\usetheme{Berlin}
\usetheme{Ilmenau}
\usecolortheme{seahorse}

%\usepackage[utf8]{inputenc}
%\usepackage{default}
%\usepackage[italian]{babel}

%\usepackage{titleref}
%\usepackage{zref-titleref}

\usepackage{amsmath}
\usepackage{amssymb}
\usepackage{amsthm}
\usepackage{colonequals}
\usepackage{mathtools}
\usepackage{xfrac}
\usepackage[all]{xy}
\usepackage{mathtools}
\usepackage{graphicx}
%\usepackage{fullpage}
\usepackage{hyperref}
\usepackage[utf8x]{inputenc}
\usepackage[italian]{babel}
\usepackage{mathtools}
\usepackage{colortbl}
\usepackage{multirow}

%\usepackage{pdftricks}
%\begin{psinputs}
%   \usepackage[pdf]{pstricks}
 %  \usepackage{multido}
%\end{psinputs}

\usepackage{ulem}

\usepackage{tikz}

%\setlength{\parindent}{0in}

\newcounter{counter1}

\theoremstyle{plain}
\newtheorem{myteo}[counter1]{Teorema}
\newtheorem{mylem}[counter1]{Lemma}
\newtheorem{mypro}[counter1]{Proposizione}
\newtheorem{mycor}[counter1]{Corollario}
\newtheorem*{myteo*}{Teorema}
\newtheorem*{mylem*}{Lemma}
\newtheorem*{mypro*}{Proposizione}
\newtheorem*{mycor*}{Corollario}

\theoremstyle{definition}
\newtheorem{mydef}[counter1]{Definizione}
\newtheorem{myes}[counter1]{Esempio}
\newtheorem{myex}[counter1]{Esercizio}
\newtheorem*{mydef*}{Definizione}
\newtheorem*{myes*}{Esempio}
\newtheorem*{myex*}{Esercizio}

\theoremstyle{remark}
\newtheorem{mynot}[counter1]{Nota}
\newtheorem{myoss}[counter1]{Osservazione}
\newtheorem*{mynot*}{Nota}
\newtheorem*{myoss*}{Osservazione}

\newcommand{\obar}[1]{\overline{#1}}
\newcommand{\ubar}[1]{\underline{#1}}

\newcommand{\set}[1]{\left\{#1\right\}}
\newcommand{\pa}[1]{\left(#1\right)}
\newcommand{\ang}[1]{\left<#1\right>}
\newcommand{\bra}[1]{\left[#1\right]}
\newcommand{\abs}[1]{\left|#1\right|}
\newcommand{\norm}[1]{\left\|#1\right\|}
\newcommand{\ceil}[1]{\left\lceil#1\right\rceil}
\newcommand{\floor}[1]{\left\lfloor#1\right\rfloor}

\newcommand{\pfrac}[2]{\pa{\frac{#1}{#2}}}
\newcommand{\bfrac}[2]{\bra{\frac{#1}{#2}}}
\newcommand{\psfrac}[2]{\pa{\sfrac{#1}{#2}}}
\newcommand{\bsfrac}[2]{\bra{\sfrac{#1}{#2}}}

\newcommand{\der}[2]{\frac{\partial #1}{\partial #2}}
\newcommand{\pder}[2]{\pfrac{\partial #1}{\partial #2}}
\newcommand{\sder}[2]{\sfrac{\partial #1}{\partial #2}}
\newcommand{\psder}[2]{\psfrac{\partial #1}{\partial #2}}

\newcommand{\intl}{\int \limits}

\DeclareMathOperator{\de}{d}
\DeclareMathOperator{\id}{Id}
\DeclareMathOperator{\len}{len}

\DeclareMathOperator{\gl}{GL}
\DeclareMathOperator{\aff}{Aff}
\DeclareMathOperator{\isom}{Isom}

\DeclareMathOperator{\im}{Im}

\newcommand*{\eqdef}{\ensuremath{\overset{\mathclap{\text{\tiny def}}}{=}}}

\begin{document}


\title{Applied $\pi$ calculus}
%\subtitle{}
%\author{Enrico Polesel}
%\institute[Scuola Normale Superiore]{Scuola Normale Superiore}
\date{\today}

\author{Enrico Polesel}



\begin{frame}[plain]
  \titlepage
\end{frame}

\begin{frame}[plain]
 \frametitle{Indice}
 \tableofcontents
\end{frame}


%\AtBeginSection[]
%{
%  \begin{frame}{\secname}
%    \tableofcontents[currentsection]
%  \end{frame}
%}


\AtBeginSubsection[]
{
  \begin{frame}[plain]{\secname $\rightarrow$ \subsecname}
    \tableofcontents[currentsubsection]
  \end{frame}
}

\section{$\pi$ calcolo e spi calcolo}

\subsection{$\pi$ calcolo}

\begin{frame}
  \frametitle{Sintassi}
  Costruiamo un $\pi$ calcolo che supporti interi e coppie, ma non
  supportiamo la scelta non deterministica ($P+Q$).

  I termini sono:
  \begin{align*}
    L,M,N,T,U,V & \coloncolonequals & \text{terms} \\
                & a,b,c,\dots,k,\dots,m,n,\dots,s & \text{name} \\
                & x,y,z & \text{variable} \\
                & (M,N) & \text{pair} \\
                & 0 & \text{zero} \\
                & suc(M) & \text{successor}
  \end{align*}
\end{frame}

\begin{frame}
  I processi sono:
  \begin{align*}
    P,Q,R & \coloncolonequals & \text{plain processes} \\
          & \mathbf{0} & \text{null process} \\
          & P | Q & \text{parallel composition} \\
          & !P & \text{replication} \\
          & \nu n.P & \text{name restriction (new)} \\
          & if\ M=N\ then\ P\ else\ Q & \text{conditional} \\
          & N(x).P & \text{message input} \\
          & \obar{N}\left< M\right> .P & \text{message output} \\
          & let\ (x,y) = M\ in\ P & \text{pair splitting} \\
          & case\ M\ of\ 0: P\ suc(x): Q & \text{integer case}
  \end{align*}
\end{frame}

\begin{frame}{Nomi privati}
  Come nel CCS, possiamo restringere i nomi ad alcuni contesti, ad
  esempio nel processo:
  \begin{align*}
  K(M) = \nu n. \pa{ \obar{n}\ang{M}.\mathbf{0} | n(x).P } | C & &  n\not\in
    fv(P)
  \end{align*}
  
  abbiamo che $C$ non pu\`o osservare il contenuto di $M$ se non per gli effetti
  di $P$, cioè:
  \[ P\set{\sfrac{M}{x}} \approx P\set{\sfrac{M'}{x}} \Rightarrow K(M)
    \approx K(M') \]
  
  Oltre alla riservatezza abbiamo anche l'autenticit\`a.
\end{frame}

\begin{frame}
  Ma, a differenza del CCS, i nomi privati possono essere comunicati
  all'esterno (scope extrusion), ad esempio:
  \begin{align*}
    A(M) & = \nu
           c_{AB}. \pa{\obar{c_{AS}}\ang{c_{AB}}. \obar{c_{AB}}\ang{M}}
    \\
    B & = c_{SB}(x). x(y) .P \\
    S & = c_{AS}(x) . \obar{c_{SB}} \ang{x} \\
    Inst(M) & = \nu c_{AS}. \nu c_{SB}. \pa{ A(M) | S | B}
  \end{align*}
  \begin{center}
    \begin{tikzpicture}[mnode/.style={circle,fill=blue!20}]
      \node[mnode] (A) at (0,0) {A};
      \node[mnode] (S) at (2,1) {S};
      \node[mnode] (B) at (4,0) {B};
      
      \draw[->] (A) -- node[above] {$c_{AS}$} (S) ;
      \draw[->] (S) -- node[above] {$c_{SB}$} (B) ;
      \draw[->] (A) -- node[below] {$c_{AB}$} (B) ;
    \end{tikzpicture}
  \end{center}
\end{frame}

\begin{frame}
  \begin{align*}
    A(M) & = \nu
           c_{AB}. \pa{\obar{c_{AS}}\ang{c_{AB}}. \obar{c_{AB}}\ang{M}}
    \\
    B & = c_{SB}(x). x(y) .P \\
    S & = c_{AS}(x) . \obar{c_{SB}} \ang{x} \\
    Inst(M) & = \nu c_{AS}. \nu c_{SB}. \pa{ A(M) | S | B} \\
    B_{spec} & = c_{SB}(x). x(y). \pa{ P\set{\sfrac{M}{y}}} \\ 
    Inst_{spec}(M) & = \nu c_{AS}. \nu c_{SB}. \pa{ A(M) | S | B_{spec}}
  \end{align*}
  \begin{itemize}
  \item \textbf{Riservatezza:} $\forall M,M'\; P\set{\sfrac{M}{y}}
    \approx P\set{\sfrac{M'}{y}} \Rightarrow Inst(M) \approx Inst(M')$;
  \item \textbf{Integrit\`a:} $\forall M \; Inst(M) \approx
    Inst_{spec}(M)$.
  \end{itemize}
\end{frame}

\begin{frame}{Limiti del $\pi$-calcolo}
  Il $\pi$-calcolo ci permette quindi di esprimere
  \begin{itemize}
  \item Segreti condivisi: $\nu k ( A|B)$;
  \item canali ristretti.
  \end{itemize}
  I nostri termini esprimono i numeri naturali e le
  coppie. 
  \vfill

  Non abbiamo modi semplici di modellare, ad esempio, una
  comunicazione cifrata (con un segreto condiviso $k$) su un canale
  pubblico.
\end{frame}

\subsection{spi calcolo}

\begin{frame}{Sintassi}
  Consideriamo un'estensione con cifratura simmetrica:
  \begin{align*}
    L,M,N,T,U,V & \coloncolonequals & \text{terms} \\
                & \cdots & \text{vedi precedenti} \\
                & \set{M}_N & \text{codifica}
  \end{align*}
  \begin{align*}
    P,Q,R & \coloncolonequals & \text{plain processes} \\
          & \cdots & \text{vedi precedenti} \\
          & case\ L\; of\; \set{x}_N\ in\ P & \text{decodifica}
  \end{align*}

  Dove
  \[ \pa{ case\ \set{M}_N\; of\; \set{x}_N\ in\ P} \approx
    P\set{\sfrac{M}{x}} \]
\end{frame}

\begin{frame}{Ipotesi su $\set{M}_N$}
  $\set{M}_N$ pu\`o essere un qualsiasi algoritmo di cifratura simmetrica,
  noi ci limiteremo a chiedere che:
  \begin{itemize}
  \item $\set{M}_N$ si possa decifrare solo conoscendo $N$;
  \item conoscere $\set{M}_N$ e $M$ non sia sufficiente per ricavare $N$;
  \item $\set{M}_N$ contenga abbastanza informazioni per capire se \`e stato
    decifrato con la chiave giusta.
  \end{itemize}
  \vfill

  Perdiamo visibilit\`a degli aspetti probabilistici e di
  complessit\`a. Ad esempio non vediamo gli attacchi del compleanno.
\end{frame}

\begin{frame}{Esempio}
  \begin{align*}
    A(M) & = \nu k_{AB}. \pa{
           \obar{c_{AS}}\ang{\set{k_{AB}}_{k_{AS}}}. \obar{c_{AB}}\ang{\set{M}_{k_{AB}}}}
    \\
    S & = c_{AS}(x).case\ x\ of\ \set{y}_{k_{AS}}in\ 
        \obar{c_{SB}}\ang{\set{y}_{k_{SB}}} \\ 
    B & = c_{SB}(x).case\ x\ of\ \set{y}_{k_{SB}}in\ c_{AB}(z). case\
        z\ of\ \set{w}_y in\ P(w) \\ 
    Inst(M) & = \nu k_{AS}. \nu k_{SB}. \pa{ A(M) | S | B } \\
    B_{spec} & = c_{SB}(x).case\ x\ of\ \set{y}_{k_{SB}}in\ 
               c_{AB}(z). case\ z\ of\ \set{w}_y in\ P(M) \\
    Inst_{s}(M) & = \nu k_{AS}. \nu k_{SB}. \pa{ A(M) | S | B_{spec}}
  \end{align*}
\end{frame}


\begin{frame}{Bisumulazione}
  
\end{frame}


\section{Applied $\pi$ calculus}

\subsection{Sintassi}

\begin{frame}{Termini}
  
\end{frame}

\begin{frame}{Processi}
  
\end{frame}

\begin{frame}{Processi estesi}
  
\end{frame}

\begin{frame}{Frame}
  
\end{frame}

\begin{frame}{fv,bv e dom}
  
\end{frame}

\subsection{Semantica}

\begin{frame}{Equivalenza struttruale}
  $\equiv$
\end{frame}

\begin{frame}{Semantica}
  
\end{frame}


\section{Equivalenze}

\subsection{Osservazionale}

\begin{frame}{Definizione}
  
\end{frame}

\begin{frame}{Esempio}
  
\end{frame}

\subsection{Statica}

\begin{frame}{Definizione}
  
\end{frame}

\begin{frame}{Esempio}
  
\end{frame}

\begin{frame}{Relazione con la osservazionale}
  
\end{frame}

\subsection{Etichettata}

\begin{frame}{Semantica etichettata}
  
\end{frame}

\begin{frame}{Definizione}
  
\end{frame}

\begin{frame}{Equivalenza con l'osservazionale}
  
\end{frame}

\section{Esempi}

\begin{frame}
  
\end{frame}

\end{document}







