\documentclass[a4paper,12pt]{article}

\usepackage{amsmath}
\usepackage{amssymb}
\usepackage{amsthm}
\usepackage{colonequals}
\usepackage{mathtools}
\usepackage{xfrac}
\usepackage{graphicx}
\usepackage{hyperref}
\usepackage[utf8x]{inputenc}
\usepackage[italian]{babel}

%\setlength{\parindent}{0in}
\usepackage{fullpage}

\theoremstyle{plain}
\newtheorem{myteo}{Teorema}[section]
\newtheorem{mylem}{Lemma}[section]
\newtheorem{mypro}{Proposizione}[section]
\newtheorem*{myteo*}{Teorema}
\newtheorem*{mylem*}{Lemma}
\newtheorem*{mypro*}{Proposizione}

\theoremstyle{definition}
\newtheorem{mydef}{Definizione}[section]
\newtheorem{myes}{Esempio}[section]
\newtheorem*{mydef*}{Definizione}
\newtheorem*{myes*}{Esempio}

\theoremstyle{remark}
\newtheorem{mynot}{Nota}[section]
\newtheorem{myoss}{Osservazione}[section]
\newtheorem*{mynot*}{Nota}
\newtheorem*{myoss*}{Osservazione}

\newcommand{\obar}[1]{\overline{#1}}
\newcommand{\ubar}[1]{\underline{#1}}

\newcommand{\set}[1]{\left\{#1\right\}}
\newcommand{\pa}[1]{\left(#1\right)}
\newcommand{\ang}[1]{\left<#1\right>}
\newcommand{\bra}[1]{\left[#1\right]}
\newcommand{\abs}[1]{\left|#1\right|}
\newcommand{\norm}[1]{\left\|#1\right\|}

\newcommand{\pfrac}[2]{\pa{\frac{#1}{#2}}}
\newcommand{\bfrac}[2]{\bra{\frac{#1}{#2}}}
\newcommand{\psfrac}[2]{\pa{\sfrac{#1}{#2}}}
\newcommand{\bsfrac}[2]{\bra{\sfrac{#1}{#2}}}

\newcommand{\der}[2]{\frac{\partial #1}{\partial #2}}
\newcommand{\pder}[2]{\pfrac{\partial #1}{\partial #2}}
\newcommand{\sder}[2]{\sfrac{\partial #1}{\partial #2}}
\newcommand{\psder}[2]{\psfrac{\partial #1}{\partial #2}}

\newcommand{\intl}{\int \limits}

\DeclareMathOperator{\de}{d}
\DeclareMathOperator{\id}{Id}

\newcommand*{\eqdef}{\ensuremath{\overset{\mathclap{\text{\tiny def}}}{=}}}

\title{Appunti seminario STT}
\author{Enrico Polesel}
\date{\today}

\begin{document}
\maketitle


\section{Sintassi}

La \textit{signature} $\Sigma$ \`e un insieme finito di simboli di
funzione e il loro numero di argomenti.

\begin{align*}
  L,M,N.T,U,V & \coloncolonequals & \text{terms} \\
  & a,b,c,\dots,k,\dots,m,n,\dots,s & \text{name} \\
  & x,y,z & \text{variable} \\
  & f(M_1,\dots,M_l) & \text{function application}
\end{align*}
con $f\in\Sigma$ e $l$ opportuno.

ground term: non contiene variabili (ma può contenere nomi e
costanti).

$u,v,w$: metavariabili per nomi e variabili.

\begin{align*}
  P,Q,R & \coloncolonequals & \text{plain processes} \\
  & \mathbf{0} & \text{null process} \\
  & P | Q & \text{parallel composition} \\
  & !P & \text{replication} \\
  & \nu n.P & \text{name restriction (new)} \\
  & if\ M=N\ then\ P\ else\ Q & \text{conditional} \\
  & N(x).P & \text{message input} \\
  & \obar{N}\left< M\right> .P & \text{message output}
\end{align*}

\begin{align*}
  A,B,C & \coloncolonequals & \text{extended processes} \\
  & P & \text{palin process} \\
  & A | B & \text{parallel composition} \\
  & \nu n.A & \text{name restriction} \\
  & \nu x.A & \text{variable restriction} \\
  & \left\{ \sfrac{M}{x} \right\} & \text{active substitution}
\end{align*}

La sostituzione $\left\{ \sfrac{M}{x} \right\}$ \`e pu\`o agire in
tutti i contesti con cui viene a contatto, per controllare questa cosa
possiamo mettere una restrizione sulla variabile $\nu x.\left( \left\{
    \sfrac{M}{x} \right\} | P\right)$.

Sostituzioni multiple e vettori:
\[ \sigma, \left\{ \sfrac{M_1}{x_1},\dots,\sfrac{M_l}{x_l} \right\},
  \left\{ \sfrac{\tilde M}{\tilde x} \right\} \]

usiamo cycle-free.

$x\sigma$: immagine di $x$ in $\sigma$, $T\sigma$: risultato
dell'applicazione di $\sigma$ alle variabili libere di $T$.

\begin{align*}
  fv(x) & \eqdef \set{x} \\
  fv(n) & \eqdef \emptyset \\
  fv\pa{f\pa{ M_1, \dots, M_l}} & \eqdef fv\pa{M_1} \cup \cdots \cup
                                  fv\pa{M_l} \\
%  fv(\mathbf{0}) & \eqdef \emptyset \\
%  fv(P|Q) & \eqdef fv(P) \cup fv(Q) \\
%  fv(!P) & \eqdef fv(P) \\
%  fv(\nu n.P) & \eqdef fv(P) \\
  & \vdots \\
  fv(N(x).P) & \eqdef fv(N) \cup \pa{fv(P) \setminus \set{x}} \\
  & \vdots \\
  fv(\nu x.A) & \eqdef fv(A) \setminus \set{x} \\
  fv\pa{ \set{ \sfrac{M}{x} } } & \eqdef fv(M) \cup \set{x}\\
  \\
  fn(x) & \eqdef \emptyset \\
  fn(n) & \eqdef \set{n} \\
  & \vdots \\
  fn(\nu n.P) & \eqdef fn(P)\setminus \set{n} \\
  fn(N(x).P) & \eqdef fn(N) \cup fn(P) \\
  fn\pa{ \set{ \sfrac{M}{x} } } & \eqdef fv(M) \\
  \\
  dom(P) & \eqdef \emptyset \\
  dom(A|B) & \eqdef dom(A) \cup dom(B) \\
  dom(\nu n.A) & \eqdef(A) \\
  dom(\nu x.A) & \eqdef(A) \setminus \set{x}\\
  dom\pa{ \set{ \sfrac{M}{x}}} & \eqdef \set{x}           
\end{align*}

Supponiamo le $\sigma$ cycle-free, le sostituzioni uniche (cioè che se
ho $A|B$ allora $dom(A) \cap dom(B) = \emptyset$) e che $x\in dom(A)$
se $\nu x.A$.

$A$ \`e chiuso sse $dom(A) = fv(A)$.

Frame ($\varphi$,$\psi$): un extended process costrutio da
$\mathbf{0}$ e sostituzioni $\set{\sfrac{M}{x}}$.

Dato un processo $A$ posso costruire $A \rightarrow \varphi(A)$
togliendo tutto ciò che non è una sostituzione attiva. Questo è il
comportamento \textbf{statico}.


Sort system per tipare bene (vedi fig.2) terms e processes. Ogni
variabile e ogni nome ha un sort.

\section{Semantica operazionale}

La nostra semantica si basa su una relazione di equivalenza $\equiv$.

Equipaggiamo la signature $\Sigma$ con una teoria equazionale, per
esempio generata partendo da assiomi o rewrite rules. Avrò delle
relazioni del tipo $\Sigma \vdash M=N$.

Contex: un'espressione con un buco. Evaluation context: un contex il
cui buco non è soggetto a replicazione, condizionale, input o
output. Il contesto $E[\_]$ chiude $A$ se $E[A]$ \`e chiuso.

Equivalenza strutturale $\equiv$: minima relazione che soddisfa le
opportune regole (vedi articolo).

Ogni extended process chiuso può essere riscritto come
\[ A \equiv \nu \tilde n . \pa{ \set{\sfrac{\tilde M}{ \tilde x}} | P
  } \]
dove $P$ \`e un plain process chiuso.

In particolare ogni frame pu\`o essere riscritto come $\varphi \equiv
\nu \tilde n. \set{ \sfrac{\tilde M}{\tilde x}}$ con $fv(\tilde M) =
\emptyset$ e $\set{\tilde n} \subseteq (\tilde M)$. L'insieme
$\set{\tilde x}$ \`e il dominio di $x$.

L'internal reduction $\rightarrow$ \`e la pi\`u piccola relazione che
soddisfa le tre opportune regole. La pi\`u interessante \`e la COMM:
\[ \bar N \ang{x}.P | N(x).Q \rightarrow P|Q \]
infatti se $x\not\in fv(N) \cup fv(M) \cup fv(P)$:
\begin{align*}
  \bar{N}\ang{M}.P | N(x).Q & \equiv \nu x.\pa{ \set{\sfrac{M}{x}} | \bar
                              N\ang{x}.P | N(x).Q} \\
  & \rightarrow \nu x .\pa{\set{\sfrac{M}{x}}|P|Q} \\
  & \equiv P|Q\set{\sfrac{M}{x}}
\end{align*}

\section{Equivalenze}

\subsection{Observational equivalence}

Diciamo che $A \Downarrow a$ se $A$ pu\`o inviare un messaggio sul
canale $a$, cio\`e $A \rightarrow ^* \equiv E\bra{\bar a \ang{M} .P}$
per qualche contesto $E$ che non lega $a$.

\begin{mydef}[Observational bisimulation]
  Una relazione simmetrica $\mathcal{R}$ fra processi estesi chiusi
  con lo stesso dominio tale che $A \mathcal{R} B$ implica:
  \begin{enumerate}
  \item $A \Downarrow a \Rightarrow B \Downarrow a$;
  \item $A' \text{closed}, A \rightarrow ^* A' \Rightarrow \exists B'\
    B \rightarrow ^* B' \text{ with } A' \mathcal{R} B'$;
  \item $\forall E[\_]$ contesto di valutazione chiuso $E[A]
    \mathcal{R} E[B]$.
  \end{enumerate}
\end{mydef}

L'equivalenza osservazionale $\approx$ \`e la pi\`u grande
bisimulazione osservazionale.

In generale non \`e decidibile.

\subsection{Static Equivalence}

\begin{mydef}[Term equivalency in frames]
  Due termini $M,N$ sono equivalenti nel frame $\varphi$ se e solo se:
  \begin{itemize}
  \item $fv(M) \cup fv(N) \subseteq dom(\varphi)$
  \item $\varphi \equiv \nu \tilde n. \sigma$
  \item $M\sigma = N \sigma$
  \item $\set{\tilde n} \cap \pa{ fn(M) \cup fn(N)} = \emptyset$.
  \end{itemize}
  per degli opportuni nomi $\tilde n$ e sostituzioni $\sigma$. Si
  scrive $\pa{M=N}\phi$.
\end{mydef}

\begin{mydef}[Equivalenza statica]
  Due frame chiusi $\varphi, \psi$ sono staticamente equivalenti
  ($\varphi \approx _s \psi$) quando 
  \begin{itemize}
  \item $dom(\varphi) = dom(\psi)$;
  \item $\forall M,N$ termini $(M=N)\varphi \Leftrightarrow (M=N)\psi$.
  \end{itemize}
  Due processi estesi chius sono staticamente equivalenti ($A \approx
  _s B$) se i loro frame sono staticamente equivalenti.
\end{mydef}

\begin{mylem}
  Dati $A,B$ processi estesi chiusi.
  \begin{itemize}
  \item $A \equiv B \vee A \rightarrow B \Rightarrow A \approx _s B$;
  \item $A \approx _s B \rightarrow E[A] \approx _s E[B]$ per ogni
    contesto di valutazione chiuso $E[\_]$.
  \end{itemize}
\end{mylem}

\begin{mypro}
  L'equivalenza osservazionale coincide con quella statica sui frame,
  ma \`e pi\`u restrittiva sui processi:
  \begin{itemize}
  \item $\varphi \approx \psi \Leftrightarrow \varphi \approx _s \psi$
  \item $A \approx B \Rightarrow A \approx _s B$ cio\`e $\approx
    \subset \approx _s$.
  \end{itemize}
\end{mypro}

Se $A \approx B$ allora $A|C$ ha gli stessi barb di $B|C$ per ogni $C$
tale che $fv(C) \subseteq dom(A)$. Allora per testare l'equivalenza
statica ci basta considerare i $C$ nella forma $if\ M=N\ then\ \bar
a\ang{n}$ dove $a$ non appartiene a $A$ o $B$.

\subsection{Labelled equivalence}

Aggiungiamo alla semantica $\rightarrow$ dei label ($A
\xrightarrow{\alpha} A'$) del tipo:
\begin{itemize}
\item $N(M)$: un input di $M$ nel canale $N$
\item $\nu x. \bar N \ang{x}$ dove $x$ \`e una variabile che non
  occorre in $N$: un output di $x$ sul canale $N$.
\end{itemize}

Nel secondo caso la variabile $x$ \`e legata al label, quindi
definiamo anche $bv(\alpha)$ nel modo ovvio. Le $fv$ sono le unioni
delle $fv$ di $M$ e $N$.

Aggiungiamo 5 regole di semantica (vedi articolo).

\begin{mydef}[Bisimulazione etichettata]
  Una labelled bisimulation \`e una relazione simmetrica $\mathcal{R}$
  sui processi estesi chiusi tale che $A \mathcal{R} B$ implica:
  \begin{enumerate}
  \item $A \approx _s B$;
  \item se $A \rightarrow A'$ con $A'$ chiuso, allora $B \rightarrow
    ^* B'$ e $A' \mathcal{R} B'$ per un qualche $B'$;
  \item se $A \xrightarrow{\alpha} A'$ con $A'$ chiuso e $fv(\alpha)
    \subseteq dom(\alpha)$ allora $B\rightarrow ^*
    \xrightarrow{\alpha} \rightarrow ^* B'$ e $A' \mathcal{R} B'$ per
    un qualche $B'$.
  \end{enumerate}
  Una labelled bisimilarity $\approx _l$ \`e la pi\`u grande labelled
  bisimulation.
\end{mydef}

\begin{myteo}
  L'equivalenza osservazionale \`e l'equivalenza etichettata: $\approx
  = \approx _l$.
\end{myteo}
% \begin{myoss}
%   Il teorema vale nell'ipotesi che le variabili esportate non siano di
%   tipo Channel.
% \end{myoss}
% \begin{myes}
%   I processi $\nu a.\pa{ \set{ \sfrac{a}{x}}}$ e $\nu
%   a. \pa{\set{\sfrac{a}{x}} | \bar a\ang{N}}$ non sono
%   osservazionalmente equivalenti ma lo sono nella semantica
%   etichettata.
% \end{myes}

\begin{mylem}{Alias elimination}
  Siano $A,B$ processi estesi chiusi, sia $M$ un termine tale che
  $fv(M) \subseteq dom(A)$ e $x$ una variabile tale che $x \not\in
  dom(A)$. Allora
  \[ A \approx _l B \Leftrightarrow \set{\sfrac{M}{x}}|A \approx _l
    \set{\sfrac{M}{x}}|B \]
\end{mylem}

\begin{mylem}[Name disclosure]
  Siano $A.B$ processi estesi chiusi e $x$ una variabile tale che $x
  \not\in dom(A)$. Allora
  \[ \nu n. \pa{ \set{\sfrac{n}{x}}|A} \approx _l \nu n. \pa{
      \set{\sfrac{n}{x}}|B } \]  
\end{mylem}

Possiamo anche rendere pi\`u espliciti di label, ad esempio creando
label del tipo $\nu y.\bar a \ang{ \pa{\text{Header},y}}$

\begin{mydef}
  Le variabili $\tilde x$ si risolvoo a $\tilde M$ in $A$ se e solo se
  \[ A \equiv \set{\sfrac{\tilde M}{\tilde x}} | \nu \tilde x.A \]
\end{mydef}

In generale dato un processo $A$ e delle variabili $\tilde x \in
dom(A)$ esistono $\tilde n, \tilde M, A'$ tali che possiamo scrivere
\[ A \equiv \nu \tilde n. \pa{ \set{\sfrac{\tilde M}{\tilde x}}
    |A'} \]
se le variabili $\tilde x$ risolvono a $\tilde M$ in $A$ allora
$\tilde n$ pu\`o essere scelto vuoto, cio\`e non abbiamo restrizioni
sui termini di $\tilde M$.

\begin{mylem}
  Le seguenti propriet\`a sono equivalenti
  \begin{enumerate}
  \item le variabili $\tilde x$ risolvono a $\tilde M$ in $A$;
  \item esiste un $A'$ tale che $A \equiv \set{\sfrac{\tilde M}{\tilde
        x}} |A'$;
  \item $\pa{\tilde x = \tilde M}\varphi (A)$ e la sostituzione
    $\set{\sfrac{\tilde M}{\tilde x}}$ \`e aciclica.
  \end{enumerate}
\end{mylem}

\begin{mylem}
  Se $A \approx _s B$ e $\tilde x$ risolve a $\tilde M$ in $A$ allora
  $\tilde x$ risolve a $\tilde M$ in $B$.
\end{mylem}

Usando questo concetto possiamo definire altre regole di sementica
etichettata e definire una nuova relazione di equivalenza $\approx _L$
che si dimostra coincidere con $\approx _l$.



\end{document}

